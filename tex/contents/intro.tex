\chapter{Introduction}
\label{intro}

\textit{This chapter presents the overall theme of this body of work. Firstly, some context is given about the overall theme of this body of work and the motivation behind it. Then, the objectives for dissertation are presented to the reader. Finally, at the end of the chapter, some information regarding the content of each chapter is presented.}


\section{Water Supply Systems}
\label{c1:s:water-supply-systems}

The water supply systems that are prevalent in our society play a very important role in our daily lives, distributing water throughout the country from water reservoirs or water treatment plants up until it reaches our houses and industries. These WSSs can be quite complex and difficult to manage without proper processes that ensure the operation of such networks is made without problems, in an environmental and economically sustainable way. For this reason, the use of specialized software to aid operators or even automatically control the operation of these WSSs is of uttermost importance nowadays. Water, be it in quantity and quality, has been a staple of all major human civilizations throughout History, from ancient roman aqueducts to the current era. 

Moving large quantities of water through enormous WSSs requires the use of vast quantities of mechanical work, which in turn requires lots of energy, namely, electric energy. With the ever-growing political, economic and environmental pressure to improve and optimize how we use energy, and with the current geopolitical issues, access to energy is getting more expensive and regulated. This means that the need for the optimization of pumping operations to reduce costs and, potentially lower energy use as well, is growing within Water Companies.



\subsection{Options}
\label{c1:ss:options}

The following options are supported:

\begin{itemize}

\item
\texttt{oldLogo}: to use the old logo of University of Aveiro.

\item
\texttt{newLogo}: to use the new logo of University of Aveiro (default behavior).

\item
\texttt{MAP}: for MAP joint doctoral programmes. The logos from the three universities (Aveiro, Minho, Porto) are used.

\item
\texttt{draft}: it prints ``DOCUMENTO PROVISÓRIO'' in the first two front pages.

\item
\texttt{draftPT}: same as \texttt{draft}.

\item
\texttt{draftEN}: same as \texttt{draft}, but instead it prints ``DRAFT DOCUMENT''.

\item
As of May 29, 2021, the department name shall not appear in the cover and the first page (top header).
A new option, \texttt{NODEPT} (no department), was created to suppress the department name (now this is the default behavior).\\
However, formerly the department name would appear in the cover and first page, therefore the old options were kept for the sake of preservation.
Any department name can be shown by using one of the following options: \texttt{DAO}, \texttt{DBIO}, \texttt{DCM}, \texttt{DCSPT}, \texttt{DECA}, \texttt{DECIVIL}, \texttt{DEGEIT}, \texttt{DEM}, \texttt{DEMAC}, \texttt{DEP}, \texttt{DETI}, \texttt{DFIS}, \texttt{DGEO}, \texttt{DLC}, \texttt{DMAT}, \texttt{DQ}.

\item
The color of the top bar, in the cover page, is defined by specifying one of the following scientific areas: \texttt{accounting}, \texttt{arts}, \texttt{economy}, \texttt{education}, \texttt{engineering}, \texttt{health}, \texttt{humanities}, \texttt{sciences}.

\end{itemize}
