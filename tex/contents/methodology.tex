\chapter{Methodology}\label{methodology}


Each Water Utility is a Client, and the Product is the Application (the \gls{dss}) that the Company (SCUBIC) provides as a cloud-based service. In this chapter, the old architecture of the Application is explained, its flaws are exposed and possible solutions are analyzed. This allows a new, proposed, architecture to be created, one that corrects those flaws.

\section{The Application}\label{methodology:s:the-aplication}

The software that the Company provides to each Client allows the Client's water operators and managers --- the Users --- to access multiple application modules:

\begin{itemize}
    \item Monitoring Module, where data from the Client's sensors can be consulted using charts and other visualization methods. This is the base module, necessary for using the other modules.
    \item Forecasting Module, which performs machine learning operations using the Client's historical sensor data and forecasted weather data to forecast water consumption for the next 24 hours after execution.
    \item Forecasting Model Training Module, which trains the machine-learning models used in the Forecasting Module.
    \item Optimization Module, which relies on the Client's sensor data, forecasted water consumption data and the Simulation submodule to optimize the pump operation schedule for lower operational cost. By optimizing the pumping operation, water and energy usage efficiency increase, lowering CO\textsuperscript{2} emissions and reducing the cost to operate the pumps.
    \item Simulation Module, where a \textit{Smart Digital Twin} of the Client's water network is created and its water pump operations simulated.
    \item \gls{kpi} Module, which performs arithmetic calculations to generate \gls{kpi} regarding the Client's operation.
    \item Solar Forecasting Module, which forecasts photovoltaic solar panel power production for use in conjunction with the Optimization Module.
\end{itemize}  
    
By using the Application's Monitoring Module, each User can access the data generated by the modules (if hired by the Client). 

For these modules to work, each Client is required to send their sensor data, with adequate frequency. Depending on the sensors, this data can be collected by the Client's sensors from every minute up to every hour, which is not synonymous with the data intake frequency. There are Clients who have sensors in remote locations which log data in 15 minute intervals, but due to power constraints this data is only sent once or twice a day to the Client's central monitoring system. Thus, the distinction between data intake frequency and data frequency must be made: the first is related to the frequency with which the sensor data packages arrive at the Application and the latter, the frequency or the time interval between each point of data in the set of sensor data. While the first is important for proper scheduling of the forecasting and optimization tasks performed by the Application, the latter is crucial and needs to have a frequency of up to 1 hour. 

Once data is sent from the Client's databases and to the Application, the data is pre-processed and stored in a time-series\footnote{Series of data points indexed in time order}\label{foot:timeseries} database. The modules access this time-series data in order to perform their tasks. 

These Forecasting, Optimization and Simulation, Solar Forecasting and KPI calculation tasks are performed with a frequency ranging from 8 to 24 hours, every day. For most clients, these tasks are performed once a day, at midnight, in order to prepare the next day's operations. The duration of said tasks can vary, depending on the amount of water consumption points to forecast, the complexity of the water network or on the amount of sensor data to process. These tasks perform calculations using medium-to-large time-series datasets and utilize machine-learning algorithms or complex optimization algorithms in conjunction with water network simulation algorithms. Thus, these tasks are run asynchronously, in \textit{Python} workers.



\section{The Old Architecture}\label{methodology:s:the-old-architecture}

The old software architecture is still in use as of the date of publication of this body of work, alongside the proposed new architecture. They are both in production, with older clients using the old architecture and new clients using the new architecture. \begin{figure}[!htbp]
    \centering
    \includegraphics[width=0.90\textwidth]{img/diagrams/pdf/old-arch-overview.drawio.pdf}
    \caption[AWS VPC Overview]{The AWS VPC used, hosting the old architecture's EC2 \gls{vps}}
    \label{fig:old-arch-overview}
\end{figure}
     

\subsection{Overview}\label{methodology:ss:overview}

The old architecture is composed of a group of \gls{ec2} instances, one for each Client, inside the Company's \gls{vpc}. Inside each \gls{ec2} instance, using Docker\footnote{https://docker.com}\label{foot:docker} for container orchestration, an Application is executed for that specific Client. Each Application consists of the joint deployment of a set of services and databases, which run inside Docker containers. As can be seen in \cref{fig:old-arch-overview}, this architecture requires an \gls{ec2} instance for each Client, which is not scalable for reasons explained later in this document.


\subsubsection{\gls{vpc}}\label{methodology:sss:vpc}

These \gls{vps} are general-purpose \gls{aws} \gls{ec2} \textit{Instances}. As can be seen on the diagram presented on \Cref{fig:old-arch-overview}, these instances are deployed to the same \gls{vpc}, sharing a private network between them.

\subsubsection{Services and Components}\label{methodology:sss:services-and-components}

\begin{figure}[!htbp]
    \centering
    \includegraphics[width=0.90\textwidth]{img/diagrams/pdf/old-arch-connections.drawio.pdf}
    \caption[Old Architecture's Containers]{Old Architecture: the containers and their connections inside the Application.}
    \label{fig:old-arch-connections}
\end{figure}




Each \gls{ec2} instance runs a Docker container for each one of the following services:
 
\begin{itemize} 

\item \textbf{InfluxDB\footnote{https://www.influxdata.com/products/influxdb-overview/}\label{foot:influxdb}} (Timeseries Database)
\item \textbf{Telegraf\footnote{https://www.influxdata.com/time-series-platform/telegraf/}\label{foot:telegraf}} (Data collecting service)
\item \textbf{MongoDB\footnote{https://www.mongodb.com/}\label{foot:mongodb}} (General use, no-SQL, Document Database)
\item \textbf{Grafana\footnote{https://grafana.com/}\label{foot:grafana}} (Web platform for data visualization, the front end of the \gls{dss})
\item \textbf{Nginx\footnote{https://www.nginx.com/}\label{foot:nginx}} (Reverse proxy with \gls{https} capabilities)
\item \textbf{Let's Encrypt\footnote{https://letsencrypt.org/}\label{foot:letsencrypt}} (Automatic \gls{tls} Certificate installer, companion for the Nginx container)
\item \textbf{Web Dev} (Flask\footnote{https://flask.palletsprojects.com/en/2.1.x/}\label{foot:flask} Web platform / API for managing Workers' settings)
\item \textbf{Redis\footnote{https://redis.io/}\label{foot:redis}} (Message Queue System for queuing Worker's jobs)
\item \textbf{OpenSSH\footnote{https://www.openssh.com/}\label{foot:openssh}} (\textit{atmoz/sftp}) (\gls{ssh} Server for receiving client data through \gls{sftp})
\item \textbf{Workers} (Celery\footnote{https://docs.celeryq.dev/en/stable/}\label{foot:celery} Container running the Forecast, Simulation and Optimization \textit{Python} Algorithms as well as the \gls{kpi} Algorithms.)
\item \textbf{Workers} (\textit{Beat}) (Celery Container that periodically \textit{triggers} jobs in the Workers container)

\end{itemize}


\subsubsection{Service's Connections}

In \Cref{fig:old-arch-connections} the relations between these containers can be schematically seen. Starting on the right side, with the  Let's Encrypt and Nginx containers, these provide outside access to the Grafana and SFTP services inside the respective containers. Data from the InfluxDB database is read by the Grafana service which allows the Client's users and the company's developers to query the database and at the same time generate charts with such information. Client sensor data is sent to the SFTP server that shares the incoming files with the Telegraf service and allows it to pre-process that sensor data and proceed to the data intake into the InfluxDB database. Then, either through remote access to the Web Dev container or automatically through the Worker Beat service, tasks are sent to the celery queue (using the Redis service) and picked up by the Worker service. This Worker service then accesses the MongoDB Database to load algorithm and device configurations and the required client sensor data from the InfluxDB database before running the tasked algorithm. Data resulting from the execution of the algorithms is then sent to the InfluxDB database, to be read by the Grafana service. There are some connections that are bidirectional, such as the Web Dev to the MongoDB database which is the service used to manipulate the MongoDB database's algorithm and device configurations.


\subsubsection{Databases}\label{methodology:sss:databases}

There are two types of databases being used by this architecture: A Timeseries Database, in this case \textbf{InfluxDB}, and an additional general-purpose Document Database: \textbf{MongoDB}. Each type of database has a different role, the first one stores the Client's timeseries data such as sensor information, pump orders, predicted tank levels, etc.
The second one, the Document Database, is responsible for storing configuration settings for each worker service (optimization, simulation and forecasting), for storing electrical tariffs data and to store sensor device's configurations.

\subsubsection{Grafana}\label{methodology:sss:grafana}

This web platform allows the visualization of the Timeseries data from the \textbf{InfluxDB} database. This is an open-source platform that runs on a docker container with little to no modifications necessary. The dashboards are built using the built-in tools and allow for complex and very informative data visualization. 

\subsubsection{Telegraf}\label{methodology:sss:telegraf}

The \textbf{Telegraf} container is used to gather the files containing the raw sensor data sent from the Client to the \gls{sftp} server. Since this container shares the file upload location folder with the \gls{sftp}, through a convoluted process of storing the filename of the last file uploaded, periodically checking for the next file and file handling \textit{spaghetti} code that spans multiple files and has an enormous codebase that weighs the docker image's file size considerably. 

\subsubsection{\gls{sftp}}\label{methodology:sss:sftp}

The \gls{sftp} service here provides a secure method for the Clients to send files containing the Timeseries data to our servers, where they can be processed and turned into actionable insights by the algorithms running in the Workers container. The Client sends their public key (from a cryptographic key pair) when the project start to authenticate against this \gls{sftp} service and uploads the files to a pre-designated folder. These files are then accessed by the Telegraf container which performs the file intake process.


\subsubsection{Nginx + Let's Encrypt}\label{methodology:sss:nginxletsencrypt}

These two containers allow secure Internet access from the \gls{ec2} instance into the correct docker container IP address and port. The Client-facing services Grafana and \gls{sftp} which, respectively, provide the web interface for the \gls{dss} and client file input service are inside containers which themselves can change their internal IP inside the Docker environment. To keep the dynamic IPs in check and allow for these services to be accessed from outside the Docker environment, the Nginx container keeps track of this dynamic IP and updates its route table accordingly. This allows for any of these two containers to restart, change their IP address and still not break the routing back to the host \gls{ec2} instance, which has an \gls{eni} associated to it exclusively. This \gls{eni} is then connected, exclusively, to a single \gls{eip} to which the Clients connect, like \Cref{fig:old-arch-nginx} implies.

As for the Let's Encrypt container, this container shares a docker volume with the Nginx container and automatically and periodically maintains the \gls{tls} certificate files that the Nginx requires in order to serve the Grafana interface through \gls{https}. 
\begin{figure}[!htbp]
    \centering
    \includegraphics[width=0.90\textwidth]{img/diagrams/pdf/old-arch-nginx.drawio.pdf}
    \caption[old-arch-nginx listing]{Internet access to the Client-facing services}
    \label{fig:old-arch-nginx}
\end{figure}


\subsubsection{Redis}\label{methodology:sss:redis}

Redis is used as a message queue backend for Celery, enabling other services to send Celery tasks to a queue for asynchronous execution by the Workers.

\subsubsection{Web Dev}\label{methodology:sss:webdev}

Based on Flask, this web application serves an \gls{api} as well as serving a web page that gives developers access to algorithm configurations and the ability to push Celery tasks to the queue. This application connects directly to both databases.

\subsubsection{Workers}\label{methodology:sss:workers}

The Workers' container image is built \textit{in-house} by the development team, using a \textit{Python} Docker image as the base image, wherein all the company's algorithms lay. The \textit{forecast}, \textit{optimization} and \textit{performance analysis}/\gls{kpi} algorithms are individually linked in a Celery configuration file, which defines how each algorithm is executed in a Celery task and how that task is called. This container executes a Celery Worker that executes all Celery Tasks in the Celery task queue.

When a task is sent to the task queue, this Celery Worker who polls the task queue, picks the task up and starts executing the task as soon as possible.

There are two Workers images, the first one contains the code for all algorithms and is the one which starts the Celery worker. The other one, which is internally called Celery Beat, executes a Celery instance in \textit{Beat} mode which sends pre-configured Celery tasks to the queue. This is used to run the algorithms periodically in order to process the Client data and generate actionable insights for the Client.

These algorithms require decent amounts of computer resources, namely CPU power and RAM capacity, in order to be able to run effectively. This is a direct contrast to the remaining components of this old architecture, which see minimal Client use and are therefore less resource intensive. In terms of storage, the situation is the opposite since these algorithms use data stored within the other services: the database services.

\subsection{Issues}\label{methodology:ss:Issues}


Besides an individual \gls{ec2} instance, each Client also has an individual GitLab\footnote{https://gitlab.com/}\label{foot:gitlab} project, which is composed of several, and different Git\footnote{https://git-scm.com/}\label{foot:git} code repositories. Each GitLab project contains the following repositories:

\begin{itemize}

    \item \textbf{dbs} (Databases configurations, build files for databases' docker images, deployment scripts)
    \item \textbf{Workers} (Build files for the Workers' docker images)
    \item \textbf{DBconnectors} (Standardized code for database access)
    \item \textbf{forecast\_optimization\_api} (Code and build files for the Web Dev docker image)
    
\end{itemize}

In the \textbf{dbs} repository, build scripts for custom docker images for InfluxDB, Nginx and Telegraf can be found. Also, here reside the scripts that are used to remotely deploy docker containers to the \gls{ec2} instances as well as the \textit{docker-compose} configuration files. The GitLab \gls{cicd} pipeline that deploys the old architecture to the instances also resides here.

As for the \textbf{DBconnectors} repository, database connectors can be found. These allow offloading the code that connects to the databases from the algorithms to a separate module, which can be reused throughout the same GitLab Project and, in theory, keep the query methods consistent for both the \textbf{Workers} and \textbf{Web Dev} codebases.

In the \textbf{Workers} repository, the code for the algorithms used by the platform to perform the forecasting, optimization and KPI calculation as well as the \textbf{DBconnectors} repository linked as a submodule can be found.

In the likeness of the \textbf{Workers} repository, the \textbf{forecast\_optimization\_api} repository also imports the \textbf{DBconnectors} repository as a submodule. This \textbf{forecast\_optimization\_api} repository is where the \textit{Web Dev} container build code is situated. 

\subsubsection{Low Cohesion and High Coupling}\label{methodology:sss:low-cohesion-and-high-coupling}

This old architecture has severe problems regarding its Cohesion and Coupling.

The readers notice that, as shown both above and on \Cref{fig:old-arch-connections}, there are multiple services performing read and write operations to the InfluxDB database. Although concurrency is not a major problem, having different schemas and tag names for InfluxDB queries in different services has historically led to multiple timeseries data not being detected when querying the database when a different querying service placed the data in the database. This is due to mismanagement of repositories and git submodules, and requires additional care, planning and communication from the developer team's side. Here, having a specific service to perform pre-prepared queries, with very detailed database schemas, to which all other services would connect to query/write to the database would solve this problem. Here, cohesion is low, since code that is used to connect to the database is spread out throughout the codebase. It also means that there is \textit{major implementation coupling}.

Then, there are issues regarding \textit{temporal coupling}. When performing data intake, the timeseries database, the data intake service(Telegraf), the SFTP service and the Nginx service all need to be up and running and accessible to be able to perform data intake. If one of these services is unavailable, data that is sent from the Client is not processed when the unavailable service becomes available again. This has been something that has happened before, with some regularity, and the only way to recover the missing data is through manual data intake of the received text or sheet files that are sent to the SFTP service which is tedious and prone to errors.

Finally, there is the issue of \textit{deployment coupling}. In order to deploy a small change to any of the services, all docker containers are shutdown and restarted, which results in considerable \textit{downtime} for each Client.


\subsubsection{Replaceability}\label{methodology:sss:replaceability}

The old architecture possesses low replaceability, since it's not easy to change a service for another, such as the timeseries database or the data intake service. For example, with the exception of the data visualization service (Grafana), major refactoring of code would have to be done, since each service connects to the timeseries database in different ways, independently.

\subsubsection{Resiliency}\label{methodology:sss:resiliency}

Despite the high coupling of the application, with the old architecture using docker containers, usually the application doesn't become totally unusable if one of the service fails. The problem lies with the docker orchestrator or the EC2 instance itself. If one of these fail, then the whole application becomes instantly unavailable. If the EC2 instance becomes unresponsive, as has happened multiple times before, then a manual, full system reboot of the instance followed by a redeployment of the whole application stack is inevitable. 

\subsubsection{Deployments}\label{methodology:sss:deployments}

A deployment of the application that uses the old architecture is a tiresome and arduous affair, since the application is highly deployment coupled as stated before. After \textit{committing} a change to one of the repositories, Deployment involves \textit{tagging} the repository of code in which the change was made in order to create the necessary docker images with those tags through \gls{cicd}. In this step, a GitLab runner will pull the repository, checkout any git submodules, perform unit tests, and then start building a docker image with the code inside the whole repository. Currently, for the old architecture, this step takes around 10 minutes to complete. After the tagging of the repository of code is done, the next step is to perform the same task with the \textit{dbs} repository, where another 10 to 20 minutes of docker image building takes place. Then, after all of the docker images are built, the GitLab runner open an SSH shell to the EC2 machine of the Client, pulls the respectively tagged docker images and performs a docker-compose operation. This operation instructs the docker orchestrator to shut down all running containers, delete the volumes (with the exception of the databases) and then re-create the containers with the newer docker images. In all, this process takes around 30 minutes to complete. During this last step, the Users experience \textit{downtime} with a duration of between 1 and 4 minutes, if the deployment is successful.

If a change is to be done for all Clients simultaneously, this process needs to be repeated individually, once for each client. 

One of the faults with the older architecture is also the lack of different environments for deployment. That means that every deployment made to each Client had the very real possibility of breaking Production for that particular Client, where the faults would impact the Client's usage of the platform directly. This is a recurring event when deploying, as the algorithms are quite complex. Given the fact that some algorithms use real-time data gathered from the last one hundred (100) days, the somewhat unpredictable nature of the algorithms' execution results make the repeatability of results from day to day not trivial.
Breaking changes are also not always apparent, since some algorithms perform calculations using data generated by other algorithms and/or real-time data and such mistakes only become apparent on the following work day, after their execution. There are cases when the algorithms run perfectly during week days, but fail during the weekends (since the water consumption patterns change accordingly).

All of these mishaps lead to the creation of a staging server where changes to the platform or algorithms could be tested with real data, causing no impact to the Clients and allowing for results to be monitored for longer periods of time to ascertain system reliability. As such, a staging environment should replicate as much as possible the production environment, be it the Operating System version, it's installed packages, \textit{Python} versions, \textit{Python} packages, the data in the server, the quick-fixes applied to production, etc.
This, however, meant that a similar, staging environment \gls{ec2} instance needed to be running simultaneously with the production environment's \gls{ec2} instance, effectively doubling the infrastructure costs. Since each Client had its own \gls{ec2} instance, this approach would also be impossible to maintain. An attempted approach was to use a single \gls{ec2} machine, sized similarly to the highest performing \gls{ec2} machine used by one of the Clients, to act as a staging server for each Client at a time. Each time a major change was to be deployed to a Client, it would be first deployed during a set time to this staging server and upon success, be deployed to the Client's production server. Having multiple developers perform different deployments, for different Clients, at the same time, meant that Deployment Frequency lowered and Lead Time for Change increased as well.

\subsubsection{Testing}\label{methodology:sss:testing}

Having the components of the application so tightly coupled, means that it requires the entire application to be executed in order to properly test the entire application. This is cumbersome and forces the developers to have a local copy of the entire application, including the timeseries data. This data can have big dimensions, and the only method to test with this data is to execute a script that connects remotely through SSH to the EC2 machine and makes a copy of the timeseries' docker instance's volume. Said volume copy is then transferred back to the developer through an SSH tunnel, so that the developer can then use that volume with the timeseries' docker instance that is running locally, for testing. Besides the massive data copy, which occupies disk space in the developers' computer and results in higher data transfer costs in the AWS account, the developer is required to have hardware capable of loading and executing all services simultaneously.

\subsubsection{Scaling}\label{methodology:sss:scaling}

The contrast between the different services' computational and storage requirements is one of the major issues of the old architecture. Adequate instance sizing is essential to lower infrastructure costs with compute resources. As can be seen in \Cref{fig:caesb-cpu-usage}, the CPU average utilization is usually very low, indicating that the resources allocated to this instance are way overestimated, elevating the infrastructure costs for no reason. However, the peaks in CPU usage that can be observed in this same Figure, which are caused by the periodically-running algorithms, push this CPU usage up to levels that suggest the allocated resources are somewhat adequate for this use-case. And wherein lies one of the major issues: over a 24-hour period, the amount of time spent with very low CPU usage is visibly and significantly superior to the time spent with adequate CPU usage for the instance size. 

\begin{figure}[!htbp]
    \centering
    \fbox{\includegraphics[width=0.95\textwidth]{img/screenshots/caesb-cpu-usage.png}}
    \caption[Client CPU Usage Example]{Client's \gls{ec2} Instance average CPU usage, during a three-day period, in 5 minutes intervals.}
    \label{fig:caesb-cpu-usage}
\end{figure}
    

The \gls{ec2} instance upon which these services reside can be provisioned and sized to different computational and storage needs. However, this would mean that it would either be adequately sized for the times the workers are dormant and undersized for when the worker's algorithms are running, or oversized for most of the time and only adequately sized while running said algorithms. Unfortunately, resizing an \gls{ec2} instance requires downtime for the whole platform, since it requires the \gls{ec2} instance to be rebooted. Since this would also stop Client access to the \gls{dss} and data intake service, this option cannot be contemplated. After testing a platform implementation with an instance adequately sized for the instants when workers are dormant, it was concluded that the algorithms would either refuse to run or crash when performing resource intensive calculations due to low RAM availability. The decision was then made, to keep the platform running in oversized, and costly, \gls{ec2} instances.

 One possible solution was to split the resources based on their compute resource requirements. Having the workers on a separate \gls{ec2} instance that would be automatically and periodically provisioned and unprovisioned according to a schedule would allow the remaining services to be placed in a lower cost \gls{ec2} instance, lowering the overall infrastructure costs. However, without altering the existing architecture, this would mean that the alteration would only be the place where the Workers' docker container would be executed. Since the amount of \gls{ec2} instances is directly proportional to the amount of Clients, having two instances would duplicate the computational resources, networks connections and storage space needed to maintain the platform for all Clients. This would exacerbate the problem of limited compute resources available to our \gls{aws} account.

%\subsection{Limited Compute Resources}\label{methodology:ss:limited-compute-resources}

One of the issues with the old architecture is that the number of \gls{ec2} instances needed was directly tied to the amount of Clients, since each Client required its own instance to host the platform, generating what is called a Scalability problem. For the company's \gls{aws} account, a limit of thirty-two (32) \gls{vcpu} units (each \gls{vcpu} corresponds to a processing thread in a CPU core) was imposed by Amazon as default, which meant that the sum of \gls{ec2} instance's \gls{vcpu} units could not surpass this value. Each client requires an \gls{ec2} instance of the type \texttt{t3a.large} or \texttt{t3a.xlarge}, respectively two (2) or four (4) \gls{vcpu} units, depending on the Client's Water Network's size and complexity and contracted services. This would mean that the amount of clients was limited from sixteen (16) clients if they all used the smaller instance or down to eight (8) clients if these Clients required more resources. As can be concluded, this is a hard limit on the amount of clients that can be served simultaneously by the company, which is an obvious problem.


\subsubsection{Cost-Effectiveness}\label{methodology:sss:cost-effectiveness}

As of June 2022, the hourly price for on-demand (\textit{EC2Cost}) \texttt{t3a.large} and \texttt{t3a.xlarge} \gls{ec2} instances in the nearest AWS region (\texttt{eu-west-3}) was, respectively, \$0.085 and \$0.1699. Since each instance requires storage, and the free storage is not enough for the data, the \gls{ebs} volume for each instance was of 256 GB in size (\textit{EBS Size}). The pricing for the \texttt{gp2} \gls{ebs} volumes is \$0.116 per GB-month of provisioned storage (\textit{EBS Pricing}). 
Thus, for each Client, the total monthly cost of just the instances \textit{Client Instance Monthly Cost} is given by the formula:

\begin{equation}
    \label{eq:ec2-instance-cost}
    Client Instance Monthly Cost = ( EC2 Cost \times 24 \times  days) + (EBS Size \times  EBS Pricing )
    \end{equation}

With this information, assuming a standard month of 30 days and a \texttt{t3a.large} instance, the average cost expenditure with a single Client's \gls{ec2} machine is:

\begin{align}
    \label{eq:ec2-instance-cost-t3alarge}
    Client Instance Monthly Cost = (0.085 \times 24 \times  30) + (256 \times  0.116)\\
    = 90.896
    \end{align}

Using the same variables, but for a \texttt{t3a.xlarge} instance:

\begin{align}
    \label{eq:ec2-instance-cost-t3axlarge}
    Client Instance Monthly Cost = (0.1699 \times 24 \times  30) + (256 \times  0.116)\\
    = 152.024
\end{align}



\subsubsection{FKMs}\label{methodology:sss:fkms}

Regarding the \gls{fkm}, based on the previous issues that have been shown, by using the old architecture, the development team has suboptimal results in the metrics.

As mentioned in the Deployment issue, the difficulty with which deployments are performed forces the development team to gather numerous changes before deploying them so that the deployment can be performed less frequently, resulting in less \textit{downtime} for the client. This, inevitably, lead to both a lower \textit{Deployment Frequency} and also a bigger \textit{Lead Time for Change}.

In the Deployment issue, it's also mentioned the amount of \textit{downtime} that is to be expected when deploying changes to production. Such an amount of \textit{downtime} is unacceptable for a normal, eventless deployment. However, when failure occurs in the application using the old architecture, the \textit{Time to Restore Service} is composed of multiple amounts of time: the time it takes for the failure to be detected, the time is takes to find a mitigation for the failure or troubleshooting time, the time it takes to re-deploy the application (which is around 20 to 30 minutes in case the mitigation requires changes to the code) and the time it takes for the recently deployed services to be back online. If failure occurs during the run of a task, that task (and subsequent tasks that failed to start) will need to be run manually as well. 

Since the development team gathers numerous changes before deploying them to production, the chance of failure increases. Besides exacerbating the problem with the \textit{Time to Restore Service}, this method of deployment also increases \textit{Change Failure Rate}.



\subsection{Observability}\label{methodology:ss:observability}

One of the issues with the old architecture was the lower Observability that it provided to the Maintainers. Despite having extensive logging for each one of the services, the other two key components of Observability - metrics and tracing - were not present at any meaningful scale. Having to peruse hundreds of lines of code, filtering different services and log levels just to manually create metrics for algorithm execution time was time-consuming and tiresome. There was also no tracing put into place anywhere in the platform. To combat this, it was stipulated by the \textit{stakeholders} that the new architecture should contemplate measures to increase observability of the entire system.

\subsubsection{Alerts}\label{methodology:sss:alerts}
A consequence of the old architecture's lack of system observability, there were no useful metrics being created and store besides the ones pertaining to the algorithm result. Metrics are required in order to, having a set of thresholds for each one of them, produce alarms. Alarms automatically inform the Maintainers and \textit{stakeholders} of unexpected system behavior or catastrophic system failure in a timely manner, giving the chance for the development team to trace the cause(s) of the problem(s) before they become apparent and/or disruptive to the Clients. For some Clients, there were metrics and alarms setup based on the Tank's water level that would send messages to a Slack channel shared between the company and the respective Client, but fell into disuse.


\section{Proposed New Architecture}\label{methodology:s:proposed-new-architecture}

\begin{displayquote}
\textit{`The primary measure of success of a software system is the degree to which it meets the purpose for which it was intended' --- Bashar Nuseibeh and Steve Easterbrook }
\end{displayquote}

Next, a proposed new architecture of the Application is presented. This new architecture minimizes or resolves the issues present in the old architecture, mentioned in the previous section.

Changing from the old architecture to the new one isn't a straightforward process. Having clients who are still using the infrastructure upon which the old architecture relies doesn't allow for mistakes while doing the migration. This brings several challenges, which are compounded by the lack of a functional new web interface for the new architecture. For this migration to occur, careful planning is to be done and checked by the \textit{stakeholders} before any changes are put into production. Measures such as changing network configurations, restarting services or run benchmarks on the old infrastructure cannot not affect any Clients using the old infrastructure.

To further complicate the planned migration, during the planning and implementation phase of this project the \textit{stakeholders} required multiple changes to accommodate new Clients, which had to be applied to the new architecture. These changes and late-requests shape the decisions taken during the planning and implementation phase of the migration. For one of the new Clients, that the \textit{stakeholders} arranged while the migration was concurring, there was a dilemma: Further increase the number of Clients using the old architecture (and subsequently, old infrastructure) or risk having this new Client as test subject for the new architecture? After discussion with the \textit{stakeholders}, the development efforts were shifted from all current project to implementing the new architecture and adapting the algorithms to make use of this new architecture.


\subsection{Solving the Deployment issues}\label{methodology:ss:solving-the-deployment-issues}

In order to solve the deployment issues, one first step is to combine all of the different Applications (one for each Client), into a single Application that can serve all Clients simultaneously, as shown in \cref{fig:new-arch-single-app}. With this, applying further modifications later on to improve the old architecture is easier, since it's only one Application that requires change. Note, however, that in \cref{fig:new-arch-single-app} the application is shown inside the \gls{vpc}, with no underlying infrastructure. This is by design, since the underlying infrastructure is not relevant to this point.

\begin{figure}[!htbp]
    \centering
    \includegraphics[width=0.90\textwidth]{img/diagrams/pdf/new-arch-single-app.drawio.pdf}
    \caption[Initially Proposed Architecture]{Initially proposed architecture. All Clients access the same Application and the Application is tested in a separate Staging and Internal environment.}
    \label{fig:new-arch-single-app}
\end{figure}


With this approach, the Application can be deployed only once for all Clients, reducing the workload of the development team. By having the Clients use the same Application, bugs or failures that might have occurred in multiple clients in the old architecture can now be resolved simultaneously by a single deployment. A single Application also means that the use of different environments can now be contemplated again, seeing as there is only the need for one copy of the Application per environment. Using three environments --- Production, Staging and Testing (referred as Internal throughout the document) --- the development team can apply changes and test them before they reach Production. Developers use the Internal environment to test new code to ensure functionality. If the Application proves to be stable, it's then copied to the Staging environment, where it stays running for a set period of time so that developers can simulate User behavior and ensure that there are no problems whatsoever with the Application. Then, after staging, the Application is sent to Production.
This simple step, although not easy to implement as it requires extensive code refactoring, is one of the most important changes to solve the Deployment, Scaling and Cost-Effectiveness issues. This measure also allows for improvement in the \gls{fkm}. \textit{Deployment Frequency} increases since there's only one Application to be deployed, therefore less work is required when compared to having to deploy multiple Applications for small changes for all Clients. \textit{Lead Time for Change}, \textit{Time to Restore Service} don't alter as much, as the deployment time for a single Application hasn't changed much with this first step and neither has the time it takes to restore service. \textit{Change Failure Rate} decreases drastically by implementing this first step. By having two other deployment environments to where the changes are first introduced, the code that is deployed to the Production server has been thoroughly tested and therefore less prone to failure.

Although moving to a single Application for all of the Clients, reducing the amount of resources needed overall even with the introduction of multiple deployment environments, the Application is still monolithic. As such, Scalability, Replaceability and Resiliency issues are still not resolved.

\subsection{Solving the Scalability issues}\label{methodology:ss:solving-the-scalability-issues}

By decoupling the Applications' components, it becomes possible to individually and automatically scale the necessary components when needed. Using this approach, services which are only run periodically can also be scaled down to 0 when not in use, such as the Workers service. If each worker performs a task per day, and if task has a duration of 15 minutes, then the total amount of time that service is online daily is reduced to around 98\%, massively reducing computational resource use and effectively lowering cost as well. As explained in the subsection regarding Scaling in the previous section, different modules of the Application have different compute resource needs. By using cheap and low-powered resources that are well-suited for the continuously running services such as databases, data intake and data visualization and using more powerful and costly resources for Workers that are only used when needed. The \textit{Cost-Effectiveness} of the Application \textit{skyrockets}.


\begin{figure}[!htbp]
    \centering
    \includegraphics[width=0.90\textwidth]{img/diagrams/pdf/new-arch-scalable.drawio.pdf}
    \caption[New Scalable Architecture]{Proposed scalable architecture. The Application's services are now independently scalable.}
    \label{fig:new-arch-scalable}
\end{figure}
 


Thus, the next iteration of the architecture encompasses these changes, as can be seen in \Cref{fig:new-arch-scalable}.

Now that each component is separate, they can be ran on different infrastructure. For that reason, the Application will be running as containers as in the previous architecture with the change that these will not be orchestrated and hosted on an \gls{ec2} instance but instead on elastic infrastructure. By using AWS's \gls{ecs}\footnote{https://docs.aws.amazon.com/AmazonECS/latest/developerguide/Welcome.html\label{foot:aws-ecs}}, each individual container can be run independently of each other and be automatically orchestrated and deployed by AWS. These containers can use either On-Demand {ec2} instances, managed by AWS, or an on-demand serverless service --- Fargate\footnote{https://aws.amazon.com/fargate/ \label{foot:fargate}} --- provided by AWS. These instances can be combined to form clusters, to where these containers can be deployed. 

Using this approach, there are multiple advantages over the previous architecture:

\subsubsection{Automatic Container Orchestration}\label{methodology:sss:automatic-container-orchestration}

By leaving the orchestration of the container up to the AWS service, the services are automatically restarted in case of failure, repeated failures generate alarms and trigger actions such as reverting to the previous service version. By pushing a new service to the orchestrator, if a previous version of that service is already running and load balancing\footnote{Efficiently distributing incoming network traffic across a pool of available services\label{foot:load-balancing}} is configured, then requests to that service will only be redirected to the newer version of the service when it reaches a stable state and is proven to be healthy. After proving that the new service is healthy and stable, the old service is then shutdown.
This drastically reduces \textit{downtime} for the Clients, since the transition from the old service to the new one is seamless and instantaneous and only occurs if the new service starts and maintains a steady and healthy state for a set period of time.

Additionally, there is an option to enable Blue Green\footnote{Blue green deployment is an application release model that gradually transfers user traffic from a previous version of an app or microservice to a nearly identical new release—both of which are running in production. \label{foot:blue-green}}, which would make the transition from the old service to the new one even more seamless.

\subsubsection{Automatic Container Scaling}\label{methodology:sss:automatic-container-scaling}

When using AWS \gls{ecs}, resource usage metrics are gathered. When alarms and actions are configured, these metrics trigger automatic scaling up or down procedures for each individual containerized service depending on set thresholds. This allows for quick scaling up when sudden usage spikes occur without needing human assistance and for scaling down after said spikes to avoid extra cost.

There are two types of scaling that can be performed with the containers. Vertical scaling --- where the resources allocated to a container are modified which requires restart of the container, and Horizontal Scaling where the cardinality of containers for a service is modified. When defining a container, defining its CPU and RAM allocations is required. Those values are not dynamic and cannot be changed after a container is running, without restarting the container. Horizontal scaling can be performed automatically, whereas vertical scaling cannot. However, the same task can be performed by a similar container that is configured to use more resources, in which case, scaling vertically can be done automatically, by first scaling horizontally this more powerful container and shutting down the previous. There will be, however, two containers of the same task running simultaneously until the most recent task reaches a steady state and allows the older task to shut down.

\subsection{Replaceability}\label{methodology:sss:replaceability-fix}

As for Replaceability, the proposed architecture improves on that measurement. Using a Backend API and HTTP calls, a common interface to use between all services is created. This Backend API allows for authenticated and authorized connections to be performed between the Application's components consistently, thus ensuring that any new service or modification to an existing one doesn't require change propagation beyond that service's business and logical vertical. 

\subsection{Resiliency}\label{methodology:sss:resiliency-fix}

Since the application is split throughout different, independent services that are managed by AWS's \gls{ecs}, there are no virtual machines or \gls{ec2} instances, no Docker orchestrators that can fail.
The availability of the \gls{ecs} service provided by AWS is very high, and therefore, the risk of failure of the infrastructure is very, very low. 

If the infrastructure is highly resilient, the possible points of failure to improve would be the Application's services themselves. As mentioned previously, the Automatic Container Orchestration service that the \gls{ecs} service provides is configured to automatically restart containers in case of failure. If the failure persists, then the orchestrator alerts the DevOps team and attempts to restart the service using a previous configuration of that same service. Therefore, failures are more likely due to occur due to human error when writing the code that is going into production. However, the code that is put into production is tested and thoroughly vetted before deployment to production, which decreases even further the risk of failure.



\subsection{Temporal Coupling}\label{methodology:sss:temporal-coupling}

Despite the best attempts, failure can still occur. If a message is sent from a service to another while it's unavailable to receive it, then it's lost. If synchronization issues arise, then the temporal coupling that torments the old architecture can make a task fail to be executed. The cause for unavailability can be due to the service failing to start, fail during execution, not being reachable due to network misconfigurations, if it experiences manual scaling events or if it's unresponsive due to high amounts of requests to it. If a new architecture is to be drawn, then it needs to address this issue.

To that end, another version of the proposed architecture is presented in \cref{fig:new-arch-queues}. 

\begin{figure}[!htbp]
    \centering
    \includegraphics[width=0.90\textwidth]{img/diagrams/pdf/new-arch-queues.drawio.pdf}
    \caption[new-arch-queues listing]{new-arch-queues caption under figure}
    \label{fig:new-arch-queues}
\end{figure}


With this new version, the Temporal Coupling has been massively reduced, the Resiliency has improved and it now allows for horizontal scalability for the services more prone to require it.
By introducing queues and load balancers to the architecture, services are no longer temporally coupled and the Application can now handle interruption of some services. Messages in the queues will be delivered to the corresponding services as soon as they become available again. With this, resiliency has also improved, since the effects of the failure of a service can be minimized by having duplicate services running and a load balancer that can re-route that service's requests to an available service as can be seen in \cref{fig:load-balancing}. It's now also more resistant to load peaks, due to the introduction of queues and load balancers. When scaling is the most adequate solution, the service is scaled and the load balancer is informed and start routing the requests to better distribute the load between the pool of service workers. If horizontal scaling is not possible, too costly or the peaks are not big enough, the queuing systems ensure that the requests are attended to as soon as possible.

\begin{figure}[!htbp]
    \centering
    \includegraphics[width=0.90\textwidth]{img/diagrams/pdf/load-balancing.drawio.pdf}
    \caption[Load Balancing Example]{A Backend API service Load Balancer re-routing traffic towards the most recent and healthiest version of the service.}
    \label{fig:load-balancing}
\end{figure}

 

\subsection{Implementation Coupling}\label{methodology:sss:implementation-coupling}

As mentioned previously, in the old architecture, the implementation coupling was strong mainly due to the database access methodology. With the newest architecture, this coupling is drastically lowered, since the only way to access the databases is through HTTP calls to the Backend API. This means that implementing methods or services that require the use of data in the databases is as simple as an API call. If a new service needs a different data structure output or input to/from the database, then the changes are made in the Backend API, making that new data structure available for future use by other services as well, in case they need it.

\subsection{Testing}\label{methodology:sss:testing-new}

The new architecture contemplates having different environments, so that a staggered deployment to Production can happen and testing can be performed in the Internal environment and then confirmed again the Staging environment. This facilitates testing the Application on real infrastructure. However, not all testing requires testing the whole Application, and for that, the new architecture allows for each one of the services to be run locally for testing and at the same time, use the services in the Application that is running in any of the Environments. In order to test a Worker service or a Backend API service, the developer needs only to execute the service to be tested locally and connect it to the \gls{vpc}.

\begin{figure}[!htbp]
    \centering
    \includegraphics[width=0.90\textwidth]{img/diagrams/pdf/new-arch-vpn.drawio.pdf}
    \caption[VPN Usage Example]{Example use of the VPN to test local Data Visualization Service with remote Backend API.}
    \label{fig:new-arch-vpn}
\end{figure}


This is possible due to three new services: A Discovery Service (AWS Cloud Map\footnote{https://aws.amazon.com/cloud-map/\label{foot:cloudmap}}), a private \gls{dns} server and a private \gls{vpn} server.
Services created in the \gls{ecs} service are automatically added to the Discovery Service by AWS. These services are given human-readable names that explicit the service name and the environment as well as the domain they belong to (e.g. \texttt{backendapi.internal.scubic.pt}). This name is then picked up by a private \gls{dns} server running alongside a private \gls{vpn} server. Together, these three services allow both developers and other services for seamless connection to services. If the service is scalable, then the address points not to the service itself but to the load balancer serving that address, as seen in \Cref{fig:new-arch-vpn}. This allows for reliable connection to the service.

This new approach uses the namespaces created by the AWS Cloud Map Discovery Service, and the naming scheme chosen was intuitive, as demonstrated in \cref{fig:service-naming}

\begin{figure}[!htbp]
    \centering
    \includegraphics[width=0.6\textwidth]{img/diagrams/pdf/service-naming.drawio.pdf}
    \caption[Service Naming Scheme]{A service naming scheme example.}
    \label{fig:service-naming}
\end{figure}


\section{New components}\label{methodology:ss:new-components-new-arch}

\subsection{Serverless}\label{methodology:sss:serverless}

As can be seen in \Cref{fig:new-arch-scalable} and \Cref{fig:new-arch-queues}, the Clients send their data to the Data Intake. The data intake process is relatively simple: Take the Client's sensor data, correctly tag the data and send it to the Data Intake Queue, to be placed in the Timeseries Database. 
However, not all Clients are the same. Some Clients send raw text files, some send \gls{csv} text files and even Microsoft Excel files. Some files are received through e-mail, others through SFTP. 
Those who do not send files, have given secure access to their databases through their own API.
Either way, each Client has a different method to give access to their sensor data. Building a single service to deal with all possible methods of data intake is not possible and would end up a monolithic solution. Thus, the solution proposed here is to separate the data intake in general from the specific data intake from each Client, by having a service that accepts only pre-processed and tagged sensor data and inputs it into the Timeseries database, and a service for each type of data intake method. Clients using e-mail would have their data pre-processed and tagged by a different service than clients who give access to their APIs.
However, having multiple services running is costly and so, these services would need to be run only when needed. A similar solution to the one used in the whole Application would be to use containers that scale up and down at set intervals. But the time that it takes for a container to be up and running is billable and superior to the average time it takes to intake data. And so, the decision was made to use AWS' Serverless Services, namely AWS Lambda\footnote{https://aws.amazon.com/lambda/\label{foot:lambda}}. Using Lambda, the code for the data intake doesn't require to be in the form of a container, thus reducing space requirements and the time it takes for a function to be called, loaded and executed. This service is billed by the millisecond and has low compute capabilities, which make it ideal for these simple tasks.

By building and deploying several Lambda functions, one for each method of data intake, thousands of data points can be read simultaneously. However, the Timeseries database expects pre-processed data with the correct tags and measurements, so a single Data Intake Lambda Function is also created to perform the standardization and normalization of the data points and ensure the correct placement in the database. One problem that might occur is at set times, such as midnight, when most Clients send their data simultaneously. In order to avoid bottlenecks and ensure all messages are delivered, a message queue is set up between all Client data intake functions and the main Data Intake function, as shown in \cref{fig:data-intake}.

\begin{figure}[!htbp]
    \centering
    \includegraphics[width=0.90\textwidth]{img/diagrams/pdf/data-intake.drawio.pdf}
    \caption[Serverless Data Intake]{Serverless Data Intake. Client's data arrives, is pre-processed by the corresponding AWS Lambda function and sent to the Data Intake queue for further processing by a main Data Intake Lambda Function}
    \label{fig:data-intake}
\end{figure}



\subsection{Backend API}\label{methodology:ss:backendapi}
The first element of the new architecture to be researched and produced was the Backend \gls{api}. A new \gls{api} solves the problem that existed with having different methods to read and write to the databases. Using this Backend \gls{api}, each service that requires access to the database is therefore required to have authorization to access the Backend \gls{api}, which in turn reinforces security regarding database access. Having a standardized method to access the databases also allows for easier debugging, since every service uses the same\gls{api}interfaces, which can help rule out databases and the Backend\gls{api}from possible fault causes.

The old architecture had a Flask \gls{api} that served a webservice through which developers could manually tweak optimization and forecasting settings and issue tasks. This \gls{api}, however, had no security features nor any authentication in place, with its access limited only through network settings, where each developer had to manually establish an \gls{ssh} tunnel to the Client's \gls{ec2} machine in order to access said \gls{api}. Due to time constraints and limited knowledge inside the company regarding securing a Flask \gls{api}, research had to be performed in order to determine the best course of action regarding the choice of web framework for a Backend \gls{api}.

\subsubsection{SQL Database}\label{methodology:sss:sql-database}
After discussion with the \textit{stakeholders} and the \gls{cto}, the decision to pursue a more feature-rich API such as Django was taken. Besides the many authentication, authorization and overall security features that Django includes in the base installation, its main feature is its \gls{orm}. After working with Flask and the MongoDB database, Django and an SQL database such as PostgreSQL is apparently an easier option that has much more community support and the SQL database is more suited for production environments where critical systems are used, in this case: Water Utilities's data. Since there wasn't big support for MongoDB in neither the Django official modules nor its community, the decision to change to an SQL Database was made. This was a simple decision, since the data that is present in old Clients' MongoDB databases is mostly auto-generated before the deployment is finalized, each time a deployment occurs. That meant that the old Clients' data wasn't hard to place in the new database.
Using Django's \gls{orm} also means that changing data-types, changing fields associated with an object and other changes that result in database schema changes are performed automatically and each change is stored in the Django directory as a migration file. This file, when placed into the git repository of the Backend API allows for these changes to be kept and enable rollbacks in case of mishaps.

\subsection{Workers}\label{methodology:ss:workers}

As a result of trying to improve the architecture, splitting each one of the tasks --- Water Forecasting, Solar Power Forecasting, Weather Forecasting, Simulation, Optimization, Model Training and KPI Calculation into their own service and code repository was successful. With this, it gave more liberty for developers to work on different tasks without complicated \textit{git} merges and also increased the resilience of the system. Where previously, if an error occurred with the Worker container it stopped all of the tasks, they are now independently orchestrated. The only exception is the innevitable temporal coupling between the Forecasting tasks and the Optimization task, since the latter requires data from the former.

\subsection{VPN and DNS}\label{methodology:ss:vpn-and-dns}

For the VPN, the chosen implementation was to use the now popular WireGuard\footnote{https://www.wireguard.com/\label{foot:wireguard}} VPN server. This was a decision taken based on \parencite{ndss_wireguard}.
As for the DNS server, the implementation chosen was a popular choice among the Networking community --- Unbound\footnote{https://nlnetlabs.nl/projects/unbound/about/}, since it was only required due to regular failures with the AWS DNS server. After a full audit report by the Open Source Technology Improvement Fund, published online \parencite{ostif_2022}, the decision was taken to use this implementation.


\subsection{Finalized Proposed Version}\label{methodology:ss:finalized-proposed-version}