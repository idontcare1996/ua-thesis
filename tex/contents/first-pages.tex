% Title, Portuguese and English titles, and thesis year.
\newcommand{\authorname}{Carlos Manuel\newline Basílio Oliveira}
\newcommand{\englishtitle}{Towards a scalable Software Architecture for Water Utilities' Decision Support Systems}
\newcommand{\portuguesetitle}{Arquitectura de Software Escalável para Sistemas de Apoio à Decisão para Entidades Gestoras de Água}
\newcommand{\thesisyear}{2022}
% \newcommand{\authorname}{Rui Marcos\newline Brandão Antunes}
% \newcommand{\englishtitle}{\LaTeX\ template for theses at University of Aveiro}
% \newcommand{\portuguesetitle}{Template \LaTeX\ para teses na Universidade de Aveiro}
% \newcommand{\thesisyear}{2022}

% Removing the lines with \setcounter{page} in the titlepage definition
% to disable page numbering restart.
% https://tex.stackexchange.com/questions/68699/how-to-avoid-page-numbering-being-re-started-by-titlepage
% https://tex.stackexchange.com/questions/27543/what-does-the-titlepage-environment-do-and-what-are-its-benefits
% https://www.tug.org/svn/texlive/trunk/Master/texmf-dist/tex/latex/base/report.cls?view=co
\makeatletter
\if@compatibility
  \renewenvironment{titlepage}
    {%
      \if@twocolumn
        \@restonecoltrue\onecolumn
      \else
        \@restonecolfalse\newpage
      \fi
      \thispagestyle{empty}%
      % \setcounter{page}\z@
    }%
    {\if@restonecol\twocolumn \else \newpage \fi
    }
\else
  \renewenvironment{titlepage}
    {%
      \if@twocolumn
        \@restonecoltrue\onecolumn
      \else
        \@restonecolfalse\newpage
      \fi
      \thispagestyle{empty}%
      % \setcounter{page}\@ne
    }%
    {\if@restonecol\twocolumn \else \newpage \fi
     \if@twoside\else
        % \setcounter{page}\@ne
     \fi
    }
\fi
\makeatother

% First pages are numbered A, B, C, ...
% Also, this avoids wrong back references with the biblatex package.
\pagenumbering{Alph}

\begingroup
% Use Helvetica font in the first pages (according to the UA rules).
% https://tex.stackexchange.com/questions/427245/how-to-use-helvetica-font-in-online-editor
% https://www.overleaf.com/learn/latex/Font_typefaces
% In fact the TeX Gyre Heros is used because it can be used as a
% substitute for Adobe Helvetica.

\ifPDFTeX
\renewcommand{\sfdefault}{qhv}
% \renewcommand{\sfdefault}{phv}
\fi

\ifLuaTeX
% \renewfontfamily\sffamily{Arial}
% \renewfontfamily\sffamily{Arimo}
% \renewfontfamily\sffamily{Helvetica}
\renewfontfamily\sffamily{TeX Gyre Heros}
\fi

% Cover page.
\TitlePage
  \HEADER{\BAR}{\thesisyear}
  \vspace*{14mm}
  \TITLE{\authorname}{\portuguesetitle}
  \vspace*{7mm}
  \TITLE{}{\englishtitle}
\EndTitlePage

% Empty page.
\titlepage\ \endtitlepage

% Initial thesis pages.
\TitlePage
  \HEADER{}{\thesisyear}
  \vspace*{14mm}
  \TITLE{\authorname}{\portuguesetitle}
  \vspace*{7mm}
  \TITLE{}{\englishtitle}
  \vspace*{15mm}
  \TEXT{}{Dissertação apresentada à Universidade de Aveiro para cumprimento dos requisitos necessários à obtenção do grau de Mestre em Engenharia Informática, realizada sob a orientação científica do Doutor \mbox{André Zúquete}, auxiliar do Departamento de Eletrónica, Telecomunicações e Informática da Universidade de Aveiro, e do Doutor António Gil
  D’Orey Andrade Campos (co-orientador), Professor auxiliar do Departamento de Engenharia
  Mecânica da Universidade de Aveiro.}
  \vspace*{\fill}
  \TEXT{}{This research was supported by  project grants through the Regional Operational Program of the Center Region (CENTRO2020) within project I-RETIS-WATER (CENTRO-01-0247-FEDER-069857)}
\EndTitlePage

% Empty page.
\titlepage\ \endtitlepage

\TitlePage
  \vspace*{55mm}
  \TEXT{\textbf{o júri~/~the jury\newline}}
       {}
  \TEXT{presidente~/~president}
       {\textbf{Professor Doutor Joaquim Arnaldo Carvalho Martins}\newline {\small
        Professor Catedrático da Universidade de Aveiro}}
  \vspace*{5mm}
  \TEXT{vogais~/~examiners committee}
       {\textbf{Professor Doutor André Ventura da Cruz Marnôto Zúquete}\newline {\small
        Professor Auxiliar da Universidade de Aveiro (orientador)}}
  \vspace*{5mm}
  \TEXT{}
       {\textbf{Professora Doutora Isabel Sofia Sousa Brito}\newline {\small
        Professora Coordenadora do Instituto Politécnico de Beja - Escola Superior de Tecnologia e Gestão - Departamento de Engenharia}}
\EndTitlePage

% Empty page.
\titlepage\ \endtitlepage

\TitlePage
  \vspace*{55mm}
  \TEXT{\textbf{agradecimentos}}
       {Agradeço o apoio da minha família, amigos e colegas da SCUBIC, e aos professores Zúquete e Gil Campos pela paciência e disponibilidade estes últimos anos.}
  \vspace*{5mm}
  \TEXT{\textbf{acknowledgments}}
       {I wish to thank my family, friends and coworkers at SCUBIC for the support, as well as prof. Zúquete and prof. Gil Campos for the availability and patience through these past years.}
\EndTitlePage

% Empty page.
\titlepage\ \endtitlepage

\TitlePage
  \vspace*{55mm}
  \TEXT{\textbf{Palavras-chave}}{Água, Arquitectura de Software, Sistemas de Apoio à Decisão, Entidades Gestoras de Água, Nexus Água-Energia, Tarifas Electricidade, Microserviços, Serverless}
  \vspace*{5mm}
  \TEXT{\textbf{Resumo}}{O fornecimento de água às populações é um serviço de qualquer grande sociedade, desde o início da Civilização. Hoje em dia, enormes quantidades de água são fornecidas constantemente a residências e indústrias variadas utilizando motores eléctricos acopolados a bombas de água que consomem vastas quantidades de energia eléctrica. Com o recurso a tarifas de electricidade variáveis e dinâmicas, dados em tempo real de sensores nas empresas de fornecimento de água e a modelos da rede de distribuição de água, o software da SCUBIC consegue monitorizar e prever consumos de água e assim optimizar a operação destas bombas por forma a baixar os custos operacionais das empresas gestoras de água.\newline O software desenvolvido pela SCUBIC permite um conjunto de serviços construídos numa fase embrionária da empresa que, por se manterem inalterados ao longo dos anos, não se adequam ao plano de negócios e aumento de requisitos por parte dos \textit{stakeholders}. Daqui surge então a necessidade de construir uma nova arquitectura de software capaz de responder aos novos desafios numa indústria cada vez mais instrumentalizada e evoluída como a da Gestão de Água.\newline Recorrendo a métodos de engenharia de software, migração de arquitecturas de software e planeamento cuidadoso, sugere-se neste trabalho uma nova arquitectura de software baseada em micro-serviços e \textit{serverless}.Esta arquitectura foi então avaliada de acordo com os índices chave de DevOps e comparada com a solução antiga. Após rever os resultados gerados pelos indicadores de performance, conclui-se que a migração foi foi benéfica para os objectivos propostos.}
\EndTitlePage

% Empty page.
\titlepage\ \endtitlepage

\TitlePage
  \vspace*{55mm}
  \TEXT{\textbf{Keywords}}{Water, Software Architecture, Decision Support Systems, Water Utilities, Water-Energy Nexus, Energy Tariffs, Microservices, Serverless}
  \vspace*{5mm}
  \TEXT{\textbf{Abstract}}{Water Supply is a staple of all civilizations throughout History. Nowadays, huge amounts of water are constantly supplied to homes and businesses, requiring the use of electric pumps which consume vast amounts of electric energy. \newline By using variable and dynamic electric tariffs, multiple real-time sensor date from Water Utilities and Water Network Modelling, the SCUBIC software is able to monitor the water networks, predict water consumption and optimize pump operation allowing the Water Utilities to lower operational costs.\newline Built during an earlier phase of the company, the SCUBIC software is a monolithic amalgamation of  services, full of compromises that cannot fulfill the latest requirements from the \textit{stakeholders} and business plan. \newline Therefore, a need to build a more modular and scalable software architecture for this software becomes apparent. Using careful planning, software engineering knowledge and literature regarding software architecture migration, a new software architecture was implemented. Results from comparisons between the older and newer architectures prove that the migration was a success and complies with the requirements set at the beginning of the project.}
\EndTitlePage

% Empty page.
\titlepage\ \endtitlepage

% End of Helvetica font.
\endgroup

% Specifying header content. In this case it only shows chapter
% information: left position at even pages, right position at odd pages.
\setlength\headheight{16pt}
\pagestyle{fancy}
\fancyhf{}
\fancyhead[LO,RE]{\fontsize{12}{14.4}}
\fancyhead[LE,RO]{\fontsize{12}{14.4}\textsc{\nouppercase{\leftmark}}}

% To change the font size of the page numbering in all pages.
\fancyfoot[C]{\small\thepage}

% From the "fancyhdr" package documentation.
% "Some LATEX commands, like \chapter, use the \thispagestyle command to
% automatically switch to the plain page style, thus ignoring the page
% style currently in effect."

% The "fancyhdr" packages does not apply same header/footer on chapter
% and non-chapter pages.
% https://tex.stackexchange.com/questions/117328/fancyhdr-does-not-apply-same-header-footer-on-chapter-and-non-chapter-pages

\fancypagestyle{plain}{%
  % Clear all header and footer fields.
  \fancyhf{}
  % Except the center.
  \fancyfoot[C]{\small\thepage}
  \renewcommand{\headrulewidth}{0pt}
  \renewcommand{\footrulewidth}{0pt}
}

% To specify 1x or 1.5x vertical spacing between lines.
% \singlespacing
\onehalfspacing

% Tables of contents, of figures, of tables.

% To count the following pages with roman numbering.
\pagenumbering{roman}

\tableofcontents
\cleardoublepage

\listoffigures
\cleardoublepage

\listoftables
\cleardoublepage

\begingroup
% To locally reduce vertical space between entries.
\setlist{itemsep=0pt,topsep=0pt,parsep=0pt,partopsep=0pt}
\printnoidxglossary
\endgroup
\cleardoublepage

% To specify 1x or 1.5x vertical spacing between lines.
% \singlespacing
\onehalfspacing

% The chapters.

% To count the following pages with Arabic numerals.
\pagenumbering{arabic}
