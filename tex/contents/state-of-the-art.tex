\chapter{State-of-the-Art}\label{state-of-the-art}

\section{Cloud Computing}\label{state-of-the-art:s:cloud-computing}

Cloud Computing is a robust and scalable dynamic platform where configurable compute resources are made available as service over normal Internet access \Parencite{alnumay_2020}.
It can also be understood, as the U.S. \gls{nist} indicates as:
\textit{Cloud computing is a model for enabling ubiquitous, convenient, on-demand network access to a shared pool of configurable computing resources (e.g., networks, servers, storage, applications, and services) that can be rapidly provisioned and released with minimal management effort or service provider interaction} \Parencite{mell_grance_2011}. 

Incorrectly used as a synonym of on-demand computing, grid computing or even \gls{saas} \Parencite{kim_2009}, Cloud Computing has become prevalent nowadays in academic, household and business environments, from small business to large enterprises \parencite{rezaei_chiew_lee_shams_aliee_2014}.

Nowadays, with the proliferation of faster Internet connections and the ever-growing landscape of Cloud Computing
\Parencite{dillon_tharam_and_wu_chen_and_chang_elizabeth}, software has become more accessible to companies than before.
By hosting and serving software through the use of Cloud Computing, that software's clients reduce both \gls{capex} and \gls{opex} by eliminating the need to buy and maintain the software and underlying infrastructure \Parencite{alnumay_2020}.

\section{Software-as-a-Service}\label{state-of-the-art:s:software-as-a-service}

\gls{saas} allows users to access software and its data, usually hosted on cloud computing services, through thin clients and/or web browsers \Parencite{mell_grance_2011}\parencite{Ali_Abdulrazzaq_and_Md_Sultan_Abu_Bakar_and_abdul_ghani_Abdul_azim_and_Zulzalil_Hazura}. The \textit{Multi-tenancy} design structure of \gls{saas} enables the software to serve multiple users (tenants), from a central server. This design allows for more efficient use of both computer resources and the human resources needed to maintain and manage them, which lowers expenditures and is therefore an imperative for businesses. Through the use of \gls{saas} instead of traditional software, the Clients no longer require the arduous task of deploying software to each one of the users, no longer dealing with varied user endpoint hardware configurations.
Cloud Computing enables ubiquitous access to \gls{saas}, which in turn makes its adoption by businesses more enticing to them. The use of \gls{saas} allows not only for lower \gls{capex} and \gls{opex} for the Client, but also enables faster and more frequent updates of the \gls{saas} (since the software manufacturer has control over it), which increases safety and security for the Client's day-to-day operations \Parencite{cavusoglu_cavusoglu_zhang_2008}. 

\subsection{Security}\label{state-of-the-art:ss:security}

There have been multiple occasions where security breaches could be prevented had the victims been using up-to-date software \Parencite{glenn_2018}. The amount of time and resources needed for patching security vulnerabilities varies from company to company but overall, can reach averages of 38 days \Parencite{rapid7_2018}. By relying on the \gls{saas} provider to patch vulnerabilities in a timely manner, Clients no longer need to allocate costly human resources to this task, which lowers expenses and human error \Parencite{glenn_2018}. 

The use of a \gls{saas} solution enables the users to access the software and its data without the use of complex networking such as \gls{vpn} and locked-down user endpoints, which have shown its faults when not properly managed, during the SARS-CoV-2 pandemic \Parencite{adams_al_shahery_chmiel_cunliffe_day_fay_gardner_giuliani_goddard_karl_2022}. By moving the majority of the responsibility for the system's security to the \gls{saas} provider who are more likely to employ security best-practices, it eliminates security threats posed by the Client's deficient security measures.


\section{Software Engineering}\label{state-of-the-art:s:software-engineering}

\subsection{Defining Requirements}\label{state-of-the-art:ss:defining-requirements}

Here, we will show what's the state-of-the-art regarding requirements definition. This is an important step in Software Engineering, or any Engineering.

Deciding on what and how to develop software is a difficult part of the software development cycle \parencite{pacheco_garcía_reyes_2018}.

\subsubsection{Identifying Key Stakeholders}\label{state-of-the-art:sss:identifying-key-stakeholders}

Before requirements elicitation, one of the most important steps is asking from whom should such requirements be elicited from. This step is crucial to prevent functional (and financial) success of the project about to be started \parencite{lewellen_2020}. Requirements elicitation should be performed during the early stages of the software planning phase in order to prevent



\section{Software Architecture}\label{state-of-the-art:s:software-architecture}

(( Que tipos de arquitecturas de software existem ))

\section{Code Deployment}\label{state-of-the-art:s:code-deployment}

\subsection{DevOps}\label{state-of-the-art:ss:devops}

The \textit{portmanteau} of Development and Operations - DevOps - is as "(...) a set of practices intended to reduce the time between committing a change to a system and the change being placed into normal production, while ensuring high quality", according to \parencite{bass_weber_zhu_2015}. There are, therefore, three things to retain from this definition. The two time periods where Development and Deployment occur, the quality of the changes to be committed to a system and the quality of the processes of putting those changes into production.

\subsubsection{Development Time}
This time






Sallin et.al. \parencite{sallin_kropp_anslow_quilty_meier_2021}

\section{Cloud-Based}\label{state-of-the-art:s:cloud-based}

\subsection{Evaluating Architectures}\label{state-of-the-art:ss:evaluating-architectures}

