\chapter{State-of-the-Art}\label{state-of-the-art}

\section{Software Engineering}\label{state-of-the-art:s:software-engineering}

\subsection{Defining Requirements}\label{state-of-the-art:ss:defining-requirements}

Here, we will show what's the state-of-the-art regarding requirements definition. This is an important step in Software Engineering, or any Engineering.

Deciding on what and how to develop software is a difficult part of the software development cycle \parencite{pacheco_garcía_reyes_2018}.

\subsubsection{Identifying Key Stakeholders}\label{state-of-the-art:sss:identifying-key-stakeholders}

Before requirements elicitation, one of the most important steps is asking from whom should such requirements be elicited from. This step is crucial to prevent functional (and financial) success of the project about to be started \parencite{lewellen_2020}. Requirements elicitation should be performed during the early stages of the software planning phase in order to prevent



\section{Software Architecture}\label{state-of-the-art:s:software-architecture}

\section{Code Deployment}\label{state-of-the-art:s:code-deployment}

\subsection{DevOps}\label{state-of-the-art:ss:devops}

The \textit{portmanteau} of Development and Operations - DevOps - is as "(...) a set of practices intended to reduce the time between committing a change to a system and the change being placed into normal production, while ensuring high quality", according to \parencite{bass_weber_zhu_2015}. There are, therefore, three things to retain from this definition. The two time periods where Development and Deployment occur, the quality of the changes to be committed to a system and the quality of the processes of putting those changes into production.

\subsubsection{Development Time}
This time 






Sallin et.al. \parencite{sallin_kropp_anslow_quilty_meier_2021}

\section{Cloud-Based}\label{state-of-the-art:s:cloud-based}